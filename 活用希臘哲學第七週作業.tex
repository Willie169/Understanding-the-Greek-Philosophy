\documentclass[a4paper,12pt]{article}
\setcounter{secnumdepth}{5}
\setcounter{tocdepth}{3}
\newcounter{ZhRenew}
\setcounter{ZhRenew}{1}
\newcounter{SectionLanguage}
\setcounter{SectionLanguage}{1}
\input{/usr/share/LaTeX-ToolKit/template.tex}
\begin{document}
\title{活用希臘哲學第六週作業}
\author{沈威宇}
\date{\temtoday}
\temdoc
\section*{問題一}
\subsection*{題目}
我們如何運用歸納法來得出「太陽每天都從東方升起」這一普遍規律?請說明其過程。
\subsection*{答案}
「太陽每天都從東方升起」的推理過程可分為以下步驟:
\ben
\item 重複的經驗觀察:人們在長時間的生活經驗中,觀察到在過去無數天的觀察中,太陽皆從東方升起,未曾出現反例。
\item 經驗的累積與概括:當觀察次數不斷增加,人們便傾向將這些個別事實概括為一個一般性陳述,即「太陽每天都從東方升起」。
\item 形成普遍規律的信念:基於經驗的高度一致性與穩定性形成結論,但仍保留被未來經驗修正的可能。
\een
\section*{問題二}
\subsection*{題目}
亞里斯多德提到了演繹推理的三段論證,請參考「所有人都用兩腳走路,蘇格拉底是人,所以蘇格拉底用兩腳走路。」來舉出一個符合此形式的實際案例,並指出演繹推理和歸納法的相異之處。
\subsection*{答案}
\subsubsection*{三段論證}
三段論證即一段符合以下形式的論證:
\bit
\item 大前提:
\[\vdash\forall x\in S\colon\varphi(x).\]
\item 小前提:
\[\vdash y\in S.\]
\item 結論:
\[\vdash\varphi(y).\]
\eit
其本質上即是全稱消去推理規則:
\[\frac{\forall x\in S\colon\varphi(x),\,y\in S}{\varphi(y)}\]
的一個應用。
\subsubsection*{符合三段論證形式的實際案例}
一個實際案例如下:
\bit
\item 大前提:所有羊膜動物都具有脊索。
\item 小前提:人是羊膜動物。
\item 結論:人具有脊索。
\eit
\subsubsection*{演繹推理與歸納法的相異之處}
演繹推理基於推理規則由普遍推理到個別,其結論在前提為真下必為真,不會被推翻。

歸納法基於經驗充分性由個別歸納成普遍,其結論超出前提內容,可能被新經驗推翻。
\end{document}
