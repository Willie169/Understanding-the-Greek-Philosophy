\documentclass[a4paper,12pt]{article}
\setcounter{secnumdepth}{5}
\setcounter{tocdepth}{3}
\newcounter{ZhRenew}
\setcounter{ZhRenew}{1}
\newcounter{SectionLanguage}
\setcounter{SectionLanguage}{1}
\input{/usr/share/LaTeX-ToolKit/template.tex}
\begin{document}
\title{活用希臘哲學第四週作業}
\author{沈威宇}
\date{\temtoday}
\temdoc
\section*{問題一}
\subsection*{題目}
普達格拉認為,人是衡量萬物之尺度。高爾吉亞認為,世界中的一切是說出來的,是受情況影響的,是情緒與傳統所塑造的,是可以經由言說而改變的。這導致了某種因人而異的相對主義。請問:對於「玉山的高度是3952公尺」這一命題,普達格拉和高爾吉亞會如何說明其相對性?你/妳認同他的觀點嗎?為什麼?
\subsection*{答案}
\subsubsection*{普達格拉的主張}
普達格拉主張「人是衡量萬物之尺度」,真理或知識的判斷取決於個人的感官經驗。對於玉山高度的判斷,如果一個測量者測得3952公尺,而另一個測量者測得3953公尺,對他們而言,各自的測量得到的命題都為真,並依賴於判斷者的認知和感受。
\subsubsection*{高爾吉亞的主張}
高爾吉亞認為語言具有強大的心理影響力,真理是言說(logos)所塑造的,並提出三個命題:「無物存在」、「即使有物存在,也無法被認識」、「即使能被認識,也無法被傳達」。「玉山3952公尺」本身是一種語言表述,它的意義取決於使用語言的人以及他們的共識,具備可塑性與相對性。
\subsubsection*{我的主張}
回顧我在第二週作業中提出的模型論真理框架,我承認普達格拉與高爾吉亞所揭示的觀察者與語言之相對性,但拒斥由此推出「不存在客觀真理」的結論。將「玉山的高度是 3952 公尺」此一命題置入該框架中,可作如下說明:
\begin{itemize}
\item \textbf{客觀可驗證真理的存在}:玉山作為自然地理對象,在一定形式語言下,具有獨立於觀察者的可推導(derivable)命題,在具有共識的形式語言與測量方法下,其高度可被反覆檢驗並證明。
\item \textbf{對普達格拉觀察者相對性的容納}:不同測量者、測量工具與測量條件等下,會產生不同的同態 $h$ 將經驗結構映射至不同的解釋結構 $\mathcal{M}$ 甚至不同的形式語言,因此觀察者相對性被保留於方法論層次,而非否定真理本身。
\item \textbf{對高爾吉亞語言相對性的容納}:此命題是一個透過特定語言形成的句子,其可理解性與意義確實依賴於共識,因此不同的形式語言 $\mathcal{L}$ 被視為真理理論組的一個可變要素,而非否定真理本身。
\end{itemize}
\end{document}
