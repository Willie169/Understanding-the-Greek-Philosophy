\documentclass[a4paper,12pt]{article}
\setcounter{secnumdepth}{5}
\setcounter{tocdepth}{3}
\newcounter{ZhRenew}
\setcounter{ZhRenew}{1}
\newcounter{SectionLanguage}
\setcounter{SectionLanguage}{1}
\input{/usr/share/LaTeX-ToolKit/template.tex}
\begin{document}
\title{活用希臘哲學第六週作業}
\author{沈威宇}
\date{\temtoday}
\temdoc
\section*{問題一}
\subsection*{題目}
柏拉圖在〈門農篇〉中,描述了蘇格拉底與小奴隸之間的故事,證明數學是人透過本能的推理得到的。對柏拉圖而言,知識不是來自外在的傳授,而是來自內在的理解,是一種「自我發覺」的過程。因此,能夠掌握知識的原因,在於人的內心世界,而不在於經驗所提供的內容。你認為單純靠天生的理性能力可以獲得知識嗎?還是知識必須建立在後天的經驗?請說明你的主張與理由,並舉例佐證你的想法。
\subsection*{答案}
我主張,人所具有的天生理性能力,本質上是證明論(proof theory)語境中一種掌握並運用推理規則(rules of inference)以生成形式證明(formal proof)的能力;然而,這種能力本身並不構成知識。

推理規則規範的是「如何從某些句子得到另一些句子」,而不是「世界是如何的」;即便這些規則可被視為先驗的(a priori),它們本身並不蘊含任何關於世界的知識內容。

知識的產生,必須結合後天經驗所提供的形式語言(formal language)與公理(axioms),在此基礎上透過推理規則進行推導,方能得到可稱為「知識」的命題。

在《門農篇》中,小奴隸並非僅僅回憶某些已存在於心中的數學真理,而是在給定定義與前提後,成功運用了推理規則,形成形式證明推導出可證命題。換言之,小奴隸展現的是「能夠產生證明」的能力,而非預先擁有證明的結論——知識。

同理,在科學中,先驗邏輯推理無法替代經驗世界的實驗結果,但能透過理性能力應用推理規則,以實驗結果推導出可證命題,即知識。
\end{document}
