\documentclass[a4paper,12pt]{article}
\setcounter{secnumdepth}{5}
\setcounter{tocdepth}{3}
\newcounter{ZhRenew}
\setcounter{ZhRenew}{1}
\newcounter{SectionLanguage}
\setcounter{SectionLanguage}{1}
\input{/usr/share/LaTeX-ToolKit/template.tex}
\begin{document}
\title{活用希臘哲學第三週作業}
\author{沈威宇}
\date{\temtoday}
\temdoc
\section*{問題一}
\subsection*{題目}
王陽明主張天下無心外之物,有人對他說:深山中的樹開花與否,似乎不關我心的事。王陽明回答:你不看此花,此花就無形象可言;你看了此花,花的顏色才反映在你心中。請問:巴門尼德對「存在」有何看法?他會認同王陽明的觀點嗎?如果會的話,請說明為甚麼。如果不會,請說明他將如何反駁王陽明。
\subsection*{答案}
\subsubsection*{巴門尼德對存在的看法}
巴門尼德將所有可以思考或談論的事物定義為「存在」(Being),並否定了「非存在」(non-Being)的存在,因為非存在既不可思考,亦不可言說。若同時相信存在與非存在,則自相矛盾,亦使知識成為不可能。巴門尼德認為,存在在其整個存在過程中保持著形式上的一致性,不生不滅,因為除了存在之外,它別無其他選擇。與之相對的,「意見」(Doxa)描述了表象世界,在這個世界中,人的感官會導向虛假和欺騙性的觀念。 
\subsubsection*{巴門尼德是否會認同王陽明?為什麼?}
不會。因為巴門尼德認為真正的存在是不生不滅的,獨立於心與感官,而王陽明心中即存在的主張,對巴門尼德而言,是將感官產生的意見誤認為真正的存在。
\subsubsection*{巴門尼德可能的反駁}
你所看到的花的開與否、顏色變化,只是感官形成的意見,它們並非真正的存在。存在本身是永恆不變的,與人的心無關。
\end{document}
